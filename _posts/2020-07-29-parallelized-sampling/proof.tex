\documentclass[fleqn]{article}
\usepackage{amsmath}
\begin{document}

Let $\mathcal{P}$ denote the probability of selecting $(r_1, ..., r_n)$ as samples, or equivalently, $(r_1, ..., r_n)$ being rows with the top $n$ largest keys, where $r_1$ has the largest key, $r_2$ has the second largest key, etc. We wish to show $\mathcal{P} = \prod\limits_{j = 1}^{n} \left( {w_j} \middle/ {\sum\limits_{k = j}^{N}{w_k}} \right)$.

\bigskip

Recall the key for the $j$-th row (denoted as $x_j$ from now on) is sampled from a probability distribution on $(-\infty, 0)$ with CDF $F_j(x) = e^{w_j \cdot x}$, and therefore the PDF of $x_j$ is $f_j(x) = F_j^\prime(x) = w_j e^{w_j \cdot x}$. Given that $x_1 \ge x_2 \ge \cdots \ge x_n$, and also, $x_n \ge x_j$ for $j \in \{ n + 1, \dots, N \}$, we have

$$
\mathcal{P} = \int_{-\infty}^{0}f_1(x_1)\int_{-\infty}^{x_1}f_2(x_2) \cdots \int_{-\infty}^{x_n}f_n(x_n) \int_{-\infty}^{x_n} f_{n + 1}(x_{n + 1}) \cdots \int_{-\infty}^{x_n} f_{N}(x_{N}) d x_N \cdots d x_2 d x_1
$$.

Working through multiple integrals like the ones above with many dots in between would be a bit too hand-wavy, so, in the interest of greater clarity, let's break down $\mathcal{P}$ into $\mathcal{P}_0$, $\mathcal{P}_1$, ..., $\mathcal{P}_n$ instead, with

\bigskip

$$
\mathcal{P}_0(x_n) = \int_{-\infty}^{x_n} f_{n + 1}(x_{n + 1}) \cdots \int_{-\infty}^{x_n} f_N(x_N) d x_N \cdots d x_{n + 1}
$$

$$
 = \prod\limits_{j = n + 1}^N\left(\int_{-\infty}^{x_n} f_j(x_j) d x_j\right)
 = \prod\limits_{j = n + 1}^N\left(F_j(x)\bigg\rvert_{-\infty}^{x_n}\right)
$$

$$
 = \prod\limits_{j = n + 1}^N\left(e^{w_j \cdot x} \bigg\rvert_{-\infty}^{x_n}\right)
 = e^{\left(\sum\limits_{j = n + 1}^N w_j\right) \cdot x_n}
$$

\bigskip

(i.e., $\mathcal{P}_0(x_n)$ is the inner-most bunch of integrals beyond $j = n$),

\bigskip

and then define

$$
\mathcal{P}_j(x_{n - j}) = \int_{-\infty}^{x_{n-j}}f_{n - j + 1}(x_{n - j + 1}) \mathcal{P}_{j - 1}(x_{n - j + 1}) d x_{n - j + 1}
$$

for $j \in \{1, \cdots, n - 1\}$, then for $j = 1$,

$$
\mathcal{P}_1(x_{n - 1}) = \int_{-\infty}^{x_{n - 1}}f_n(x_n) \mathcal{P}_0(x_n) d x_n
$$

$$
 = \int_{-\infty}^{x_{n - 1}} w_n \left. e^{w_n \cdot x_n} \right.\cdot \left[ e^{\left(\sum\limits_{k = n + 1}^N w_k\right) \cdot x_n} \right]d x_n
 = w_n\int_{-\infty}^{x_{n - 1}}e^{\left(\sum\limits_{k = n}^N w_k\right) \cdot x_n} d x_n
$$

\bigskip

$$
 = \left( w_n \middle/ {\sum\limits_{k = n}^N w_k}\right) \cdot \left [ e^{\left(\sum\limits_{k = n}^N w_k\right) \cdot x_n} \right ]\bigg\rvert_{-\infty}^{x_{n - 1}}
 = \left( w_n \middle/{\sum\limits_{k = n}^N w_k} \right ) \cdot e^{\left(\sum\limits_{k = n}^N w_k\right) \cdot x_{n - 1}}
$$

Now all that is remaining is simply an exercise of proof by mathematical induction, where given the induction hypothesis

$$
\mathcal{P}_j(x_{n - j}) = \left[ \prod\limits_{h = n - j + 1}^{n} \left( w_h \middle/ {\sum\limits_{k = h}^N w_k} \right) \right] \cdot e^{\left(\sum\limits_{j = n - j + 1}^N w_j\right) \cdot x_{n - 1}}
$$
, which is already shown to be true for $j = 1$, show it is true for $j \in \{2, \dots, n - 1\}$.



\end{document}
